\documentclass{article}

\newcommand{\cmd}[2]{\textbf{#1} 
        \hspace{0.5cm}\parbox[t]{10cm}{#2}}
\newcommand{\cmdr}[1]{{\em #1}}
\newcommand{\baag}{{\em baag}}
\newcommand{\shikaar}{{\em shikaar}}

\begin{document}

\title{Engine-UI protocol for BagChal}
\author{jaH saaymi \and Suraj Adhikari}
\maketitle

\section{Basics}

Two pipes are used for communication between the UI and engine. The
first one (stdin for engine) to send commands from the UI to the
engine. The second one (stdout for engine) to send commands from the
engine to the UI. The communication between engine and UI takes the
form of a series for commands in fullduplex mode, i.e. each entity can
be both in the process of receiving and sending commands at the same
time. Each command begins after a new line, and ends when a newline is
received. There need not be two newlines between commands. Commands
that are not understood are ignored.

\section{Commands from the UI to the engine}

\begin{description}
\item[bagchalui1]{The command is sent by the user interface to start
the communication. The engine need not send any commands before it
receives this command. But the engine may send the \cmdr{ping} command
to check whether an engine is connected; in this case, the engine will
reply with this command. After receiving this command the engine
should go to the mode supporting this protocol.}

\item[new]{Resset the board to the standard bagchal initial
position. Reset the number of \shikaar to 20. Set \shikaar on
move. Associate the engine's clock with \baag and the opponent's clock
with \shikaar. Reset the clocks and time controls with the start of a
new game. Stop clocks.}

\item[quit]{The bagchal engine should immediately exit.}

\end{description}

\end{document}