\documentclass{article}\usepackage{noweb}\pagestyle{noweb}\noweboptions{}\begin{document}\nwfilename{types.nw}\nwbegindocs{0}\nwenddocs{}\nwbegincode{1}\moddef{*}\endmoddef\nwstartdeflinemarkup\nwenddeflinemarkup% ===> this file was generated automatically by noweave --- better not edit it
  \LA{}types.h\RA{}
\nwendcode{}\nwbegindocs{2}\nwdocspar

\texttt{types.h} defines the simple data-types used in the
bagchal engine.

\nwenddocs{}\nwbegincode{3}\moddef{types.h}\endmoddef\nwstartdeflinemarkup\nwenddeflinemarkup

#ifndef bagchal_engine_types_h
#define bagchal_engine_types_h

\LA{}typedefs\RA{}

#endif

\nwendcode{}\nwbegindocs{4}\nwdocspar

The {\Tt{}i64{\_}t\nwendquote} is a 64 bit integer. Since the actual base type is different
for different compilers. We use a typedef for a 64 bit integer.

\nwenddocs{}\nwbegincode{5}\moddef{typedefs}\endmoddef\nwstartdeflinemarkup\nwenddeflinemarkup

typedef unsigned long long i64_t;

\nwendcode{}\nwbegindocs{6}\nwdocspar

The {\Tt{}i2{\_}t\nwendquote} is at least 2 bits long, but could be longer.
The {\Tt{}i1{\_}t\nwendquote} is at least 1 bit long, but could be longer.

\nwenddocs{}\nwbegincode{7}\moddef{typedefs}\plusendmoddef\nwstartdeflinemarkup\nwenddeflinemarkup

typedef char i2_t;
typedef char i1_t;

\nwendcode{}\nwbegindocs{8}\nwdocspar
The following provides a boolean datatype lacking in C.

\nwenddocs{}\nwbegincode{9}\moddef{typedefs}\plusendmoddef\nwstartdeflinemarkup\nwenddeflinemarkup

typedef i1_t Boolean_t;

#define TRUE 1
#define FALSE 0
\nwendcode{}\end{document}

